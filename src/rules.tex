\section{Cantus}
Welcome to the Cantus!
It is an evening where choreography, magic, songs, drinks and joy go hand in hand.
Although, you have to follow some rules at a Cantus it is the participants who determine the success of it!

One of the most important items at a Cantus is this songbook, which is also called \emph{Codex}.
You may keep this songbook as memory after the Cantus has come to an end.
It is tradition that participants write something on each others' songbooks. 
The very same tradition dictates that you may not read these messages before you wake up the next morning.

\subsection{Participants of the Cantus}
First we introduce the participants of the Cantus.
They are divided into two groups: 1) The Praesidium 2) The Corona.
The Praesidium are the leaders of the Cantus, the Praesidium consist of
\begin{itemize}
        \item The Senior whom is the main leader of the Cantus. The senior is always right.
        \item The punisher whom punishes misbehaving participants.
        \item The procantor whom is the lead singer.
\end{itemize}
The rest of the participants form the Corona.

\subsection{Code of Conduct}
There are five simple rules which must be obeyed at any time during the Cantus!
They are
\begin{itemize}
        \item[§1] The Senior is \textbf{always right.}
        \item[§2] In case the Senior is not right, refer to rule §1.
        \item[§3] Keep silence! It is \textbf{forbidden to talk} during the Cantus.
        \item[§4] \textbf{Do not leave} your place without permission of the Senior.
        \item[§5] \textbf{Do not applaud}, instead thump the table with your knuckles.
\end{itemize}
Anyone who breaks one or several rules will get punished!

\subsubsection*{The beginning and end of the Cantus}
Before the Cantus is started everyone stands up. Then the Senior initiates the Cantus with the words: \textbf{``Silentium, omnes ad sedes''}, after which everyone must be silent and sit down.

The Cantus ends when the Senior shouts: \textbf{``Cantus ex, Party in''}.

\subsubsection*{Enrich the Cantus with your singing voice}
The Cantus is all about singing hence it is mandatory to sing along. You are probably not the worst singer, most likely your neighbour is even worse than you are.
The lyrics for every song of the Cantus is available in this songbook thus not knowing the lyrics is \emph{not an excuse} for not singing.
If you do not sing you will get punished. Moreover, extraordinary bad singing may also get punished.

Furthermore, the Senior decides ultimately which songs are to be sung.

\subsubsection*{Addressing the Corona}
You are not allowed to speak at any time during the Cantus with \emph{one exception} though. You may ask the Senior for permission to speak by using the following procedure:
\begin{enumerate}
        \item Form a V with your hands, arms, legs, or whatever. However there must not be any ambiguity or you risk getting punished.
        \item When the Senior sees fit the Senior will point at you at which time you must say: \textbf{``Senior, verbum peto!''}
        \item The Senior may grant your request by replying \textbf{``habes''} or deny it by replying \textbf{``non-habes''}.
        \item In case you were granted the honour of speaking the Corona you will have to start your speech by addressing the Praesidium and Corona: \textbf{``Praesidium, Corona''} and finish your speech by saying \textbf{``Dixi''}.
\end{enumerate}

\subsubsection*{Breaks}
At some point there will be a general break for everyone to get some fresh air and go do their business at the toilet. The Senior decides when breaks are held and the Senior will introduce general breaks with the words \textbf{``Tempus commune''}. The general breaks last 10 minutes. The Cantus starts again with the words \textbf{``Tempus ex, Cantus in''}.

If you urgently need to go to toilet you may request permission by using the following procedure:
\begin{enumerate}
        \item Stand up and form a T with your hands, arms, legs, or whatever. However there must not be any ambiguity or you risk getting punished.
        \item When the Senior sees fit the Senior will point at you at which time you must say: \textbf{``Senior, tempus privatum peto''}.
        \item The Senior can agree either un- or conditionally by replying \textbf{``habes''} or decline your request by replying \textbf{``non habes''}.
\end{enumerate}
If the Senior agrees \emph{conditionally} then you will be given a small task to carry out before you may to go the toilet.

\subsubsection*{Silence is golden}
You have to remain silent at all times (except for when we are singing).
Whenever the Senior wants complete silence the Senior shouts \textbf{``silentium!''}. The Corona replies by shouting \textbf{``triplex''} and then keeps silent and pays attention to the Praesidium.

\subsubsection*{Drinking}
Traditionally you drink beer during a Cantus.
However if you for some reason cannot drink beer you may ask the Senior to declare you \textbf{beer impotent}. As beer impotent you are allowed to drink water during the Cantus.

You are not allowed to drink without the Senior's permission.
After a splendid performance the Senior may toast to the Corona by saying \textbf{``Prosit Corona!''} at which point the Corona stands up and replies \textbf{``Prosit Senior, prosit Corona!''}. Furthermore, the Senior will tell you how much to drink for instance:
\begin{description}
        \item[Ad fundum:] Finish your glass.
        \item[Semi ad fundum:] Drink half of your glass.
        \item[Ad libidum:] Drink as much as you like.
\end{description}
Moreover, in between some songs it may be important to clear the throat properly. When the Senior deems it necessary to clear the throat the Senior shouts: \textbf{``Skyl halsen!''} (Danish for ``clear your throat'') at which point you drink the little \emph{refresher} next to your beer.

\textbf{Hint:} Nobody can force you to drink, not even the Senior. But please make sure you control your alchohol consumption such that you do not drink beyond your own limits.

\subsubsection*{Punishments}
If you break any of the rules, or if the Senior feels like it, you will get punished. For a minor rule offense you may get punished in your seat. However for greater offenses or in the case of multiple offenses you may be summoned to \textbf{the middle} for a severe punishment. Usually, the punishments increase proportional in terms of humilation as the Cantus progresses.

The Corona may urge the Senior to impose a punishment on someone by pointing at the person and shouting \textbf{``Ad pistum''}. Then the Senior decides whether or not a punishment should be carried out.

\subsection{A final word}
Please behave and obey the rules so that we may enjoy the Cantus.
The Senior has to power to dispel any participant from the Cantus, so please remember that you are not only ruining the fun for yourself but also for the rest of the participants by misbehaving.
