\beginsong{Der Er Et Yndigt Land}[by={written by Adam Oehlenschläger},li={National antheme of Denmark}]
\beginverse
Der er et yndigt Land, 
Det står med brede Bøge 
Nær salten Østerstrand; 
Det bugter sig i Bakke, Dal, 
Det hedder gamle Danmark, 
Og det er Freias Sal.
\endverse

\beginverse
Der sad i fordums Tid 
De harniskklædte Kæmper, 
Udhvilede fra Strid; 
Saa drog de frem til Fienders Meen, 
Nu hvile deres Bene 
Bag Høiens Bautasteen.
\endverse 

\beginverse 
Det Land endnu er skiønt, 
Thi blaa sig Søen belter, 
Og Løvet staaer saa grønt; 
Og ædle Qvinder, skiønne Møer, 
Og Mænd og raske Svende 
Beboe de Danskes Øer.
\endverse

\beginverse
Vort Sprog er stærkt og blødt, 
Vor Tro er reen og luttret 
Og Modet er ei dødt. 
Og hver en Dansk er lige fri, 
Hver lyder tro sin Konge, 
Men Trældom er forbi.
\endverse
 
\beginverse
Et venligt Syd i Nord 
Er, grønne Danarige, 
Din axbeklædte Jord. 
Og Snekken gaaer sin stolte Vei. 
Hvor Ploug og Kiølen furer, 
Der svigter Haabet ei.
\endverse

\beginverse 
Vort Dannebrog er smukt, 
Det vifter hen ad Havet 
Med Flagets røde Bugt. 
Og stedse har sin Farve hvid 
Dit hellige Kors i Blodet, 
O Dannebrog, i Strid.
\endverse

\beginverse 
Karsk er den Danskes Aand, 
Den hader Fordoms Lænker, 
Og Sværmeriets Baand. 
For Venskab aaben, kold for Spot, 
Slaaer ærlig Jydes Hierte, 
For Pige, Land og Drot.
\endverse

\beginverse
Jeg bytter Danmark ei, 
For Ruslands Vinterørkner, 
For Sydens Blomstermai. 
Ei Pest og Slanger kiende vi, 
Ei Vesterlandets Tungsind, 
Ei Østens Raseri.
\endverse
 
\beginverse
Vor Tid ei staaer i Dunst, 
Den hævet har sin Stemme 
For Videnskab og Kunst. 
Ei Bragis og ei Mimers Raab 
Har vakt i lige Strækning 
Et bedre Fremtids Haab.
\endverse

\beginverse 
Ei stor, vor Fødestavn, 
Dog hæver sig blandt Stæder 
Dit stolte Kiøbenhavn. 
Til bedre By ei Havet kom, 
Ja ingen Flod i Dalen, 
Fra Trondhiem og til Rom.
\endverse

\beginverse 
Med hellig Varetægt 
Bevare du, Alfader! 
Vor gamle Kongeslægt. 
Kong Fredrik ligner Fredegod;
Hvor er en bedre Fyrste, 
Af bedre Helteblod?
\endverse

\beginverse 
Hil Drot og Fædreland! 
Hil hver en Danneborger, 
Som virker hvad han kan. 
Vort gamle Danmark skal bestaae, 
Saalænge Bøgen speiler 
Sin Top i Bølgen blaa.
\endverse
\endsong